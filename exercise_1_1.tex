\documentclass{article}
\usepackage{amssymb}
\begin{document}

\section{Empty Set as a Semigroup}
\subsection{Claim}
Let $S$ be the empty set.
Let $f$ be the binary operator on $S$.
Then $(S,f)$ forms a semigroup.
\subsection{Proof}
As $S = \emptyset$ the domain and codomain of $f$ are empty.
So it is vacuously true that $f$ is associative regarding $S$.
$\square$


\section{Groups are Semigroups}
\subsection{Claim}
Let $(G,f)$ be a group.
Then $(G,f)$ is also a semi-group.
\subsection{Proof}
By definition of a group, $f$ must be a closed, associative, binary operator. Thus the pair forms a semim-group.
$\square$


\section{Constant Map constitutes a Semigroup}
\subsection{Claim}
Let $S$ be any non-empty set.
Let the binary operator $f : S \times S \to S$ be a constant map.
Then $(S,f)$ forms a semigroup.
\subsection{Proof}
Since $f$ is a constant map, it's output is fixed.
Denote that output $c$.
When $f$ acts on any pair in $S$ it always yields $c$.
For any fixed, but arbitrary $x,y \in S$, even $f(x,c) = f(c,y) = c$.
So $f$ is trivially associtative ( and commutative ).
$\square$


\section{Singletons as Semigroups}
\subsection{Claim}
Let $S$ be a singleton $\{x\}$.
Let $f$ be the function $ f: S \times S \to S$.
Then $(S,f)$ form a semigroup.
\subsection{Proof}
Since $S$ only has one elment, $f$ must be a constant map.
Thus, from the prior exercise, $(S,f)$ forms a semigroup.
$\square$	
	
\section{Symmetric Semigroup}
\subsection{Claim}
Let $X$ be any set.
Let $\textbf{Symm}(X)$ be the set of all injections from $X$ to itself.
With function composition as the binary operator, $\textbf{Symm}(X)$ forms a semigroup.
\subsection{Proof}
Each $f, g \in \textbf{Symm}(X)$ have the same image and domain - both of which are equal to $X$.
So each $f$ is able to compose with any $g$ and vice versa.
Thus one can freely associate with any elements in the $\textbf{Symm}(X)$ - it's function composition, trivially associative as it inherits it from the set theoretic relationship $\times$ operator.
$\square$



\end{document}