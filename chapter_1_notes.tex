\documentclass{article}

\begin{document}
\section{Intro to Semigroups}
\subsection{Basic Definitions}
A \textbf{semigroup} is a set S together with an associative binary operaiton on S.



\subsection{Examples}
\begin{enumerate}
	\item Empty Set with the Empty Function as the binary operator.
	\item Groups are semigroups.
	\item Any set with a constant map for it's binary operator forms a semigroup.
	\item Singleton with only possible function.
	
	\item The Cyclic group $C_3$.
	\item The flip-flop monoid $\{$ "set", "reset", "do nothing"$\}$
	\item The set $\{-1, 0, 1\}$ under integer multiplication.
	\item The \textbf{symmetric semigroup} - For any set $X$ the mappings ( or transformation )of $X$ into $X$ are the elements of a semigroup; the operation is composition of mappings.
	\item The symmetric semigroup on a set $X$ can be expanded as follows. A \textbf{partial mapping} of $X$ into itself ( usually called a \textbf{partial transformation} of $X$ ) is a mapping $\alpha : A \to X$ whose domain $A$ is a subset of $X$. We generally write mappings on the left, but for partial transformations we use the right operator notation. When $\alpha : A \to X$ and $\beta : B \to X$ are partial transformations, the domain of $\alpha \beta $ is $ D = \{ x\in A; x\alpha \in B \}$; then $x ( \alpha \beta ) = (x \alpha ) \beta$ for all $x \in D$.
\end{enumerate}

\subsubsection{Exercises - Demonstrate Examples}
\begin{enumerate}
	\item Verify that 1-4 and 8-9 above are examples of semigroups.
	\item Given any sets $I$ and $\Lambda$ show that the operation $(i, \lambda)(j,\mu) = (i,\mu)$ on $I \times \Lambda$ is associative.
\end{enumerate}


\end{document}